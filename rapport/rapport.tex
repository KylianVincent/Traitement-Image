%%%%%%%%%%%%%%%%%%%%%%%%%%%%%%%%%%%%%%%%%%%%%%%%%%%%%%%%
% 							                   PREAMBULE        
%%%%%%%%%%%%%%%%%%%%%%%%%%%%%%%%%%%%%%%%%%%%%%%%%%%%%%%%

\documentclass[a4,12pt]{article}

%--- Packages génériques ---%

\usepackage[francais]{babel}
\usepackage[utf8]{inputenc}
\usepackage[T1]{fontenc}
\usepackage[babel=true]{csquotes}
\usepackage{amsmath}
\usepackage{amssymb}
\usepackage{float}
\usepackage{graphicx}
\usepackage{hyperref}

%--- Structure de la page ---%

\usepackage{fancyheadings}

\topmargin -1.5 cm
\oddsidemargin -0.5 cm
\evensidemargin -0.5 cm
\textwidth 17 cm
\setlength{\headwidth}{\textwidth}
\textheight 24 cm
\pagestyle{fancy}
\lhead[\fancyplain{}{\thepage}]{\fancyplain{}{\sl ENSIMAG 2A}}
\chead[\fancyplain{}{{\sl }}]{\fancyplain{}{{TP Traitement d'Image}}}
\rhead[\fancyplain{}{}]{\fancyplain{}{Loiodice \& Vincent}}
\lfoot{\fancyplain{}{}}
\cfoot{\fancyplain{}{}}
\cfoot{\thepage }
\rfoot{\fancyplain{}{}}

%--- Style de la zone de code ---%

\usepackage{tikz}
\usetikzlibrary{calc}
\usepackage[framemethod=tikz]{mdframed}
\usepackage{listings}             
\usepackage{textcomp}

\lstset{upquote=true,
        columns=flexible,
        keepspaces=true,
        breaklines,
        breakindent=0pt,
        basicstyle=\ttfamily,
        commentstyle=\color[rgb]{0,0.6,0},
        language=Scilab,
        alsoletter=\),
        }

\lstset{classoffset=0,
        keywordstyle=\color{violet!75},
        deletekeywords={zeros,disp},
        classoffset=1,
        keywordstyle=\color{cyan},
        morekeywords={zeros,disp},
        }

\lstset{extendedchars=true,
        literate={0}{{\color{brown!75}0}}1 
                 {1}{{\color{brown!75}1}}1 
                 {2}{{\color{brown!75}2}}1 
                 {3}{{\color{brown!75}3}}1 
                 {4}{{\color{brown!75}4}}1 
                 {5}{{\color{brown!75}5}}1 
                 {6}{{\color{brown!75}6}}1 
                 {7}{{\color{brown!75}7}}1 
                 {8}{{\color{brown!75}8}}1 
                 {9}{{\color{brown!75}9}}1 
                 {(}{{\color{blue!50}(}}1 
                 {)}{{\color{blue!50})}}1 
                 {[}{{\color{blue!50}[}}1 
                 {]}{{\color{blue!50}]}}1
                 {-}{{\color{gray}-}}1
                 {+}{{\color{gray}+}}1
                 {=}{{\color{gray}=}}1
                 {:}{{\color{orange!50!yellow}:}}1
                 {é}{{\'e}}1 
                 {è}{{\`e}}1 
                 {à}{{\`a}}1 
                 {ç}{{\c{c}}}1 
                 {œ}{{\oe}}1 
                 {ù}{{\`u}}1
                 {É}{{\'E}}1 
                 {È}{{\`E}}1 
                 {À}{{\`A}}1 
                 {Ç}{{\c{C}}}1 
                 {Œ}{{\OE}}1 
                 {Ê}{{\^E}}1
                 {ê}{{\^e}}1 
                 {î}{{\^i}}1 
                 {ô}{{\^o}}1 
                 {û}{{\^u}}1 
        }

%--- Raccourcis commande ---%

\newcommand{\R}{\mathbb{R}}
\newcommand{\N}{\mathbb{N}}
\newcommand{\A}{\mathbf{A}}
\newcommand{\B}{\mathbf{B}}
\newcommand{\C}{\mathbf{C}}
\newcommand{\D}{\mathbf{D}}
\newcommand{\ub}{\mathbf{u}}

%--- Mode correction et incréments automatiques ---%

\usepackage{framed}
\usepackage{ifthen}
\usepackage{comment}
\usepackage{graphicx}

\newcounter{Nbquestion}

\newcommand*\question{%
\stepcounter{Nbquestion}%
\textbf{Question \theNbquestion. }}

\newboolean{enseignant}
%\setboolean{enseignant}{true}
\setboolean{enseignant}{false}

\definecolor{shadecolor}{gray}{0.80}

\ifthenelse{
\boolean{enseignant}}{
\newenvironment{correction}{\begin{shaded}}{\end{shaded}}
}
{
\excludecomment{correction}
}

%--- Style de l'encadré des questions ---%

\mdfsetup{leftmargin=12pt}
\mdfsetup{skipabove=\topskip,skipbelow=\topskip}

\tikzset{
	warningsymbol/.style={
	rectangle,draw=red,
	fill=white,scale=1,
	overlay}}
\global\mdfdefinestyle{exampledefault}{
	hidealllines=true,leftline=true,
	innerrightmargin=0.0em,
	innerleftmargin=0.3em,
	leftmargin=0.0em,
	linecolor=red,
	backgroundcolor=orange!20,
	middlelinewidth=4pt,
	innertopmargin=\topskip,
}

\global\mdfdefinestyle{answer}{
	hidealllines=true,leftline=true,
	innerrightmargin=0.0em,
	innerleftmargin=0.3em,
	leftmargin=0.0em,
	linecolor=green,
	backgroundcolor=white,
	middlelinewidth=4pt,
	innertopmargin=\topskip,
}

%%%%%%%%%%%%%%%%%%%%%%%%%%%%%%%%%%%%%%%%%%%%%%%%%%%%%%%%
% 							               EN-TETE        
%%%%%%%%%%%%%%%%%%%%%%%%%%%%%%%%%%%%%%%%%%%%%%%%%%%%%%%%

\title{\textbf{TP1 Traitement d'Image\\Convolution : Lissage et détection de contours}}
\author{
\begin{tabular}{cc}
	\textsc{Loiodice Thomas} & \textsc{Vincent Kylian} \\
\end{tabular}}   
\date{\small \today}

\makeatletter
	\def\thetitle{\@title}
	\def\theauthor{\@author}
	\def\thedate{\@date}
\makeatother 

\usepackage{etoolbox}
\usepackage{titling}
\setlength{\droptitle}{-7em}

\setlength{\parindent}{1cm}

\makeatletter
% patch pour le bug concernant les parenthèses fermantes d'après http://tex.stackexchange.com/q/69472
\patchcmd{\lsthk@SelectCharTable}{%
  \lst@ifbreaklines\lst@Def{`)}{\lst@breakProcessOther)}\fi}{}{}{}
  
%%%%%%%%%%%%%%%%%%%%%%%%%%%%%%%%%%%%%%%%%%%%%%%%%%%%%%%%
% 							CORPS DU DOCUMENT          
%%%%%%%%%%%%%%%%%%%%%%%%%%%%%%%%%%%%%%%%%%%%%%%%%%%%%%%%

\begin{document}
\maketitle


%%%%%%%%%%%%%%%%%%%%%%%%%%%%%%%%%%%%%%%%%%%%%%%%%%%%%%%%
% 						                  	PARTIE I         
%%%%%%%%%%%%%%%%%%%%%%%%%%%%%%%%%%%%%%%%%%%%%%%%%%%%%%%%

\section{Partie I : Lissage linéaire}

\subsection{FFT et filtrage fréquentiel}


En étudiant les résultats obtenus sur les différents types de bruit nous avons vu que ce filtrage est plus adapté à certains types de bruit que d'autres. En effet le filtre semble être efficace sur les bruits gaussien et speckle mais semble moins adapté au bruit poivre et sel.\\

L'estimation d'un résultat visuel est parfois compliqué, en effet le "meilleur" lissage est difficile à choisir : plus le lissage est fort plus les couleurs sont constantes et proches des images d'origine mais plus les contours deviennent épais et flous. Ainsi dans nos estimations visuelles nous avions souvent choisi un $\sigma$ plus faible que celui maximisant le PSNR.\\


Exemple pour l'image \textit{figure2bb50.pgm} :\\

\noindent
\begin{minipage}[c]{0.20\linewidth}
	\begin{center}
		\includegraphics[width = 33mm]{./img/formes2bb50.jpg}
		\textit{origine}
	\end{center}
\end{minipage}
\begin{minipage}[c]{0.20\linewidth}
	\begin{center}
		\includegraphics[width = 33mm]{./img/2bb50-1.jpg}
		\textit{$\sigma = 1$}
	\end{center}
\end{minipage}
\begin{minipage}[c]{0.20\linewidth}
	\begin{center}
		\includegraphics[width = 33mm]{./img/2bb50-2.jpg}
		\textit{$\sigma = 2$}
	\end{center}
\end{minipage}
\begin{minipage}[c]{0.20\linewidth}
	\begin{center}
		\includegraphics[width = 33mm]{./img/2bb50-3.jpg}
		\textit{$\sigma = 3$}
	\end{center}
\end{minipage}
\begin{minipage}[c]{0.20\linewidth}
	\begin{center}
		\includegraphics[width = 33mm]{./img/2bb50-4.jpg}
		\textit{$\sigma = 4$}
	\end{center}
\end{minipage}


\subsubsection*{Bruit gaussien}
Ce bruit, caractérisé par un flou et l'addition d'un bruit blanc, nous a semblé être le plus réceptif au lissage. Avec l'image \textit{formes2bb40.pgm} on obtient les résultats suivants :
\begin{center}
\begin{minipage}[c]{0.30\linewidth}
	\begin{center}
		\includegraphics[width = 50mm]{./img/2bb40.jpg}
		\textit{origine}\\
		\textit{PSNR = 16.36}
	\end{center}
\end{minipage}
\begin{minipage}[c]{0.30\linewidth}
	\begin{center}
		\includegraphics[width = 50mm]{./img/2bb40-1.jpg}
		\textit{$\sigma = 1$}\\
		\textit{PSNR = 26.86}
	\end{center}
\end{minipage}
\begin{minipage}[c]{0.30\linewidth}
	\begin{center}
		\includegraphics[width = 50mm]{./img/2bb40-2_5.jpg}
		\textit{$\sigma = 2.5$}\\
		\textit{PSNR = 30.29}
	\end{center}
\end{minipage}
\end{center}

Le résultat obtenu permet d'obtenir un bon PSNR dès $\sigma=1$, cette valeur nous a semblé un bon compromis entre le lissage et l'apparition de flou, cependant le PSNR maximal peut être obtenu pour $\sigma=2.5$.


\subsubsection*{Bruit speckle}
Le bruit speckle semble
Nous avons pris une image qui nous a semblé proche qualitativement et quantitativement niveau bruit de l'image testée en section précédente (\textit{formes2bb40.pgm}) pour le buit speckle : l'image \textit{formes2sp1.pgm}.
\begin{center}
	\begin{minipage}[c]{0.30\linewidth}
		\begin{center}
			\includegraphics[width = 50mm]{./img/2sp1.jpg}
			\textit{origine}\\
			\textit{PSNR = 16.77}
		\end{center}
	\end{minipage}
	\begin{minipage}[c]{0.30\linewidth}
		\begin{center}
			\includegraphics[width = 50mm]{./img/2sp1-1.jpg}
			\textit{$\sigma = 1$}\\
			\textit{PSNR = 27.23}
		\end{center}
	\end{minipage}
	\begin{minipage}[c]{0.30\linewidth}
		\begin{center}
			\includegraphics[width = 50mm]{./img/2sp1-2_4.jpg}
			\textit{$\sigma = 2.4$}\\
			\textit{PSNR = 30.52}
		\end{center}
	\end{minipage}\\
\end{center}

Ici le meilleur résultat est obtenu pour $\sigma=2.4$ et les résultats obtenus sont très proches de ceux pour un bruit gaussien.


\subsubsection*{Bruit poivre et sel}
Le bruit poivre et sel est celui qui résiste le plus à notre filtrage. Étant constitué de points de forte disparité (blancs et noirs, aux extrêmes de l'échelle de couleur) face aux pixels de l'image, ceux-ci ne sont lissés dans l'image qu'au prix d'un fort flou sur les contours. Pour illustrer cela nous avons choisi l'image \textit{formes1pets5.pgm}.
\begin{center}
	\begin{minipage}[c]{0.30\linewidth}
		\begin{center}
			\includegraphics[width = 50mm]{./img/1pets5.jpg}
			\textit{origine}\\
			\textit{PSNR = 11.06}
		\end{center}
	\end{minipage}
	\begin{minipage}[c]{0.30\linewidth}
		\begin{center}
			\includegraphics[width = 50mm]{./img/1pets5-1.jpg}
			\textit{$\sigma = 1$}\\
			\textit{PSNR = 18.94}
		\end{center}
	\end{minipage}
	\begin{minipage}[c]{0.30\linewidth}
		\begin{center}
			\includegraphics[width = 50mm]{./img/1pets5-2_4.jpg}
			\textit{$\sigma = 2.4$}\\
			\textit{PSNR = 20.49}
		\end{center}
	\end{minipage}
\end{center}

L'image est ici, d'origine, bien plus bruitée que les images utilisées précédemment dans les tests mais cela montre la persistance des points de bruits poivre et sel face au filtrage.

\subsection{Convolution spatiale}

Nous avons tout d'abord observé que plus le sigma est choisi grand, plus la taille de la fenêtre doit être importante pour observer les mêmes résultats qu'avec un filtrage fréquentiel. En effet, plus la valeur du lissage $\sigma$ est grande, plus la gaussienne va être étalée et sa décroissance lente. Ainsi afin de prendre en compte tous les coefficients non nuls du filtre gaussien, il sera nécessaire de prendre en compte des pixels plus éloignés du pixel calculé, et donc d'élargir la fenêtre W.\\

Cet effet est notamment visible sur le traitement ci-dessous de l'image \textit{formes1pets5.pgm} avec $\sigma = 5$ :\\

\noindent
\begin{minipage}[c]{0.50\linewidth}
	\begin{center}
		\includegraphics[width = 70mm]{./img/2sp5.jpg}\\
		\textit{Image bruitée}\\
		\textit{$PSNR_{formes1}=10.58$}
	\end{center}
\end{minipage}
\begin{minipage}[c]{0.50\linewidth}
	\begin{center}
		\includegraphics[width = 70mm]{./img/2sp5-5-5}\\
		\textit{$\sigma=5$ et W=5}\\
		\textit{$PSNR_{formes1}=10.93$}
	\end{center}
\end{minipage}\\
\\

Sur la figure de droite où l'on a choisi une fenêtre de taille 5, l'image est clairement assombrie, les contributions des pixels éloignés mais n'étant pas annulés par la gaussienne ne sont en effet pas prises en compte. En choisissant une fenêtre plus importante ($W = 20$), on obtient un résultat équivalent au résultat obtenu par FFT :\\

\noindent
\begin{minipage}[c]{0.50\linewidth}
	\begin{center}
		\includegraphics[width = 70mm]{./img/2sp5-5-20.jpg}\\
		\textit{$\sigma=5$ et W=20}\\
		\textit{$PSNR_{formes1}=25.91$}
	\end{center}
\end{minipage}
\begin{minipage}[c]{0.50\linewidth}
	\begin{center}
		\includegraphics[width = 70mm]{./img/2sp5-5.jpg}\\
		\textit{Avec FFT, $\sigma=5$}\\
		\textit{$PSNR_{formes1}=25.91$}
	\end{center}
\end{minipage}\\
\\

Ainsi, nous avons étudié sur des exemples l'écart absolu entre le résultat obtenu par un filtrage fréquentiel et celui obtenu par un filtrage spatial en fonction de la taille de fenêtre utilisée. Nous avons choisi pour celà les images bruitées \textit{formes1pest5.pgm}, \textit{formes2sp5.pgm} \textit{formes2bb67.pgm} afin d'expérimenter les variations sur des images de contrastes et bruits différents.

Nous avons choisi des images présentant un niveau de bruit assez élevé pour permettre une bonne visualisation du filtrage.

\begin{center}
	\includegraphics[width = 170mm]{./img/1pets5sce.jpg}\\
	\textit{Comparaison sur l'image formes1pets5.pgm}
\end{center}
\vspace{1cm}

\begin{center}
	\includegraphics[width = 170mm]{./img/2sp5sce.jpg}\\
	\textit{Comparaison sur l'image formes2sp5.pgm}
\end{center}
\vspace{1cm}

\begin{center}
	\includegraphics[width = 170mm]{./img/2bb67sce.jpg}\\
	\textit{Comparaison sur l'image formes2bb67.pgm}
\end{center}

\noindent
Ainsi on peut voir que :
\begin{itemize}
	\item Il n'est pas nécessaire d'avoir un masque d'une largeur très importante, en effet dès une taille de 40 pour les quatre valeurs de $\sigma$ testées, le résultat est identique à celui obtenu par FFT.
	\item On peut déduire de ces expérimentations une loi empirique reliant $\sigma$ et $W$ :\\
	\begin{center}
		\fbox{$W(\sigma) \geq 3 \sigma$}
	\end{center}
\end{itemize}

\vspace{1em}
On peut remarquer que pour $\sigma=0.5$, et plus généralement $\sigma < 1$, on obtient un écart qui ne tend pas vers $0$ et reste à une valeur fixe, ceci est le même phénomène pour toutes les images. Nous ne parvenons pas à expliquer cet écart non présent pour les autres valeurs.

\subsection{Complexité et comparaison des 2 méthodes}
Pour la comparaison du temps de calcul nous avons utilisé l'image \textit{formes2bb50.pgm} avec les résultats précédents, soit une taille de masque telle que $W=3\sigma$.

Nous avons ainsi comparé les temps relatifs entre l'implémentation fréquentielle (en prenant en compte le surcoût de passage par la FFT, le shift et les opérations inverses) et celle spatiale (implémentée par un filtre séparable). L'écart relatif a été calculé ainsi :\\

\begin{equation}
E_{f/s}=100\frac{T_{freq}-T_{spat}}{T_{freq}}
\end{equation}

La figure obtenue, moyennée sur 5 mesures, est la suivante :
\begin{center}
	\includegraphics[width = 170mm]{./img/timeDiff.jpg}
\end{center}

On voit sur cette figure que pour de petites valeurs de $\sigma$, et donc une faible variance de gaussienne, le filtrage spatial est plus intéressant que le filtrage fréquentiel. En effet pour ces faibles tailles la taille du masque à utiliser est aussi petite, ce qui implique un nombre de calculs moindre (Le filtrage spatial est alors jusqu'à 20\% plus rapide).

Cette tendance s'inverse pour les valeurs de $\sigma$ importantes, c'est ici l'effet inverse qui se produit, le masque doit, pour ces valeurs, prendre une taille importante qui alourdit les calculs (jusqu'à 50\% plus lent que le filtrage fréquentiel pour $\sigma=10$).\\

\noindent
L'équivalence entre les deux méthodes se trouve à une taille proche de $\sigma = 2.5$. Ainsi :
\begin{itemize}
	\item Pour une taille $\sigma < 0.5$ c'est le filtrage \textit{spatial} qui est à privilégier
	\item A l'inverse, pour $\sigma \geq 0.5$ le filtrage \textit{fréquentiel} est celui qui minimise le temps de calcul
\end{itemize}


% Fin section 1 %
\section{Partie II : Détection de contours}

\subsection{Opérateurs différentiels du premier ordre}

\subsection{Opérateurs différentiels du deuxième ordre}

\subsection{Comparaison}


% Fin section 2 %
\end{document}

% Fin du document LaTeX
